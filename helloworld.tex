\chapter{Hello, World!}

To start the primer, this chapter aims to get the reader hit the ground running with several examples which elucidates the \textit{feel} of writing code in C. 

\section{Compiling your code}

When developing programs using a programming language, knowing how to execute your program is an important process. For example, when you have a Python file called \verb|hello_world.py|, you might run it by typing the following in your terminal:
\codebox{
\begin{typewriter}
    python3 hello\_world.py
\end{typewriter}
}

Python is an example of what we call an \textit{interpreted} language. In this case, we have the \textit{interpreter} (\verb|python3|) reading the user's code line-by-line and executing it directly.

On the other hand, C is a compiled language; which means that in order to run your code, we would need to compile it first into an executable, which is then run. To execute a code written in C (e.g. \verb|hello_world.c|), we first invoke the compiler:
\codebox{
    \begin{typewriter}
        gcc hello\_world.c
    \end{typewriter}
}
which creates the executable file called \verb|a.out|, which we can then run by using
\codebox{
    \begin{typewriter}
        ./a.out
    \end{typewriter}
}

Note: As it is executing the program, the computer has no idea that a.out was just created from some C program: It is simply blindly executing the code found within the a.out file, just as it blindly executes the code found within the gcc file in response to the first command. \\ % from C for Python Devs

\begin{exercise} \label{helloworld1} 
Compiling your first C program \\
Create a file called \verb|hello.c| with the following content:
\begin{minted}{c}
#include <stdio.h>
int main() {
    printf("Hello, world!\n");
}
\end{minted}
Compile and run the program.
\end{exercise}

\warningbox{
    If you would like to name the output something other than \begin{typewriter}a.out\end{typewriter}, use the \begin{typewriter}-o\end{typewriter} flag during the compilation process, e.g.
    \begin{typewriter}gcc hello\_world.c -o hello\_world\end{typewriter} would produce an executable called \begin{typewriter}hello\_world\end{typewriter}.
}


\section{Example 1: Adding two numbers}
In \autoref{helloworld1}, we have seen a program which would print out "Hello, world!". In contrast to Python, where a \begin{typewriter}print("Hello, world!")\end{typewriter} one-liner suffices, the equivalent C code contains 4 lines. Generally speaking, writing code in C tends to be more verbose than Python, with more parts which the programmer need to annotate. 

Going one step beyond the "Hello, world" program, we are going to examine another simple program. For convenience, we will put the C program and an equivalent Python program side-by-side.


\noindent
\begin{minipage}{0.40\textwidth}
    \begin{minted}[stripnl=false, linenos]{python3}
def add(x: int, y: int) -> int:
    return x + y

z = add(3, 4)
z = z + 5
print("z is %d" % z)
#print("z is {}".format(z))
#print(f"z is {z}")
    \end{minted}
\end{minipage}%
\hfill%
\begin{minipage}{0.35\textwidth}
    \begin{minted}[stripnl=false, linenos]{c}
#include <stdio.h>
int add(int, int);

int main() {
    int z = add(3, 4);
    z = z + 5;
    printf("z is %d", z);
    return 0;
}

int add(int x, int y) {
    return x + y;
}
    \end{minted}
\end{minipage}
\newpage
From this example, we observe the following:
\begin{itemize}
\item \textbf{Whitespace, Semicolons, and Braces}\\
In Python, whitespace characters like tabs and newlines are important: You separate your statements by placing them on separate lines, and you indicate the extent of a block (like the body of a while or if statement) using indentation. These uses of whitespace are idiosyncrasies of Python.

On the other hand, C does not use whitespace except for separating words. Most statements are terminated with a semicolon (\verb|;|), and blocks of statements are indicated using a set of braces, '\verb|{|' and '\verb|}|'.

\item \textbf{Variable Declaration and Assignment} \\
The first time we use the variable \verb|z| in the C example above, we have \verb|int| written preceeding it. This is an example of variable declaration: on the very first occurence of the variable, it must be declared of a certain type. Once a variable is declared, its type cannot be changed.

Line 7 of the code above is an example of a variable declaration \textit{and} assignment. We could split it into two lines like so:
\begin{minted}{c}
int z;
z = add(3, 4);
\end{minted}
where the first line is the declaration, and the second is the assignment. Note that a variable has an unknown value before its first assignment, even if it has been declared.

On the subsequent uses of \verb|z|, \verb|int| should not be written in front of it anymore (try writing \verb|int z| instead of \verb|z| on line 8, and compile the program). \textit{Declaration} of a variable should only be done once. 

\item \textbf{Function Definitions} \\
By looking at the \begin{typewriter}add()\end{typewriter} function (line 11-13), we could observe how functions are defined:
\begin{itemize}
    \item C does not have the \verb|def| keyword to define functions.
    \item Function definitions in C begins with the return type of the function, in this case \verb|int|.
    \item The function parameters \verb|x| and \verb|y| are written inside the parantheses as in Python, but it is necessary to write the type of each parameter.
\end{itemize}

Unlike Python, all code in C must be nested within functions, and functions cannot be nested within each other. This results in the overall structure of C programs being typically very straightforward: it is a list of function definitions, one after another, each containing a list of statements to be executed when the function is called.

\item \textbf{The\begin{typewriter} main()\end{typewriter} function}\\
Programs have one special function named \verb|main|, whose return type is an integer. This function is the “starting point” for the program: The computer essentially calls the program's \verb|main| function when it wants to execute the program. The return value is an integer, and in the example we returned a 0, indicating that the program has finished successfully. 


\item \textbf{The \begin{typewriter} printf()\end{typewriter} function}
The \verb|printf| function is similar to \verb|print| in Python -- both of them enables us to display reesults for users to see. 

While the \verb|print| function in Python is built-in, the \verb|printf| function is not. \verb|printf| is declared in the standard header file \verb|stdio.h|, hence the need to put include it at the start of the program. (Try to remove the line and compile + run the program; what happens?)

There are some differences between Python's \verb|print| and C's \verb|printf|. One of the biggest differences is the fact that \verb|printf| always takes in a string as the first argument, and the following parameters indicate the values to print. In effect, this looks similar to Python's "old-school" string formatting with the \% as shown in the example. 

\item \textbf{Function prototypes}\\
Since we are writing the \verb|add| function after the \verb|main| function, we need to tell the compiler on the types expected in the function. If we were to omit line 3, the compiler would emit a warning as we are using \verb|add| in line 5 but we have not seen the function definition yet. 

For this particular example, we can delete the function prototype by moving the \verb|add| function to above the \verb|main| function; however, bigger codebases often necessitates the usage of these function prototypes, especially when we are calling functions defined in another file. 

In the case of functions defined in another file, the function prototypes are usually put in header files (.h). We can see this happening with the \verb|printf| function, which has the function prototype in \verb|stdio.h|.
\end{itemize}

\newpage
\section{Example 2: Absolute Sum of Array}
Suppose that you are tasked to find the sum of the absolute value of each entry in an array (or a list, in Python). Below are the corresponding Python and C code\footnote{The indentation used throughout would be the 1TBS style to save space -- as mentioned, whitespace is not that important to the compiler! Use the indentation style that you like, but be consistent.} which would do the task.

\noindent
\begin{minipage}{0.40\textwidth}
% \begin{minted}{python3}
% arr = [1, -2, 3, -4, 5]
% res = 0
% for entry in arr:
%     if entry > 0:
%         res += entry
%     elif entry == 0:
%         continue
%     else:
%         res += (entry * -1)
% print(res)
% \end{minted}
% \rule{\textwidth}{1px}
\begin{minted}[linenos]{python3}
arr = [1, -2, 3, -4, 5]
res = 0
for i in range(len(arr)):
    if arr[i] > 0:
        res += arr[i]
    elif arr[i] == 0:
        continue
    else:
        res += (arr[i] * -1)
print(res)
\end{minted}
\end{minipage}
\hfill
\begin{minipage}{0.40\textwidth}
\begin{minted}[linenos]{c}
#include <stdio.h>

int main() {
    int arr[5] = {1, -2, 3, -4, 5};
    int len = 5;
    int res = 0;
    
    for (int i = 0; i < len; i++) {
        if (arr[i] > 0) {
            res += arr[i];
        } else if (arr[i] == 0) {
            continue;
        } else {
            res -= arr[i];
        }
    }
    printf("%d", res);
}
\end{minted}
\end{minipage}


The above Python solution is not the most 'pythonic' way to solve this task: indeed, one can use a for-each loop (e.g. \begin{typewriter}for entry in arr:\end{typewriter} in line 3) but the style above is chosen to closely follow the C example given. From this example, we can observe the following:

\begin{itemize}
    \item \textbf{Array declaration and definition} \\
    Arrays in C are akin to Lists in Python, with some key differences. First, C arrays could only contain one type of data. In the example above, line 4 declares and initializes an array of \verb|int|. Second, the array size is fixed\footnote{More experienced readers will point at VLA and / or dynamically-allocated arrays; for simplicity, we are skipping this.}; for this example, we have an array size of 5. There is no way to mutate \verb|arr| into one of size 6, unlike in Python where you can call \verb|.append()|.  

    \item \textbf{Array access} \\
    Accessing an array in C looks similar to array accesses in Python -- compare line 4 in the Python code to line 9 in the C code. Note that valid array indexes for the example is only 0 to 4 inclusive, i.e. you cannot access arr[-1] as in Python.
    
    Arrays in C have less capability than lists in Python. One of the important differences is that C lacks boundary check in array accesses. If we try to access \verb|arr[10]| in the Python code above, we will get an \verb|IndexError|, since the length of arr is 5. However, if we write \verb|arr[10]| in C, it will still compile and run, giving the programmer gibberish results and potentially crashing the program on running. 
    
    This should tell you that it's important to keep track of the length of the arrays you have in C. Unfortunately, there is no equivalent of Python's \verb|len()| function to check the length of the array. The programmer needs to keep another variable, in the example above the \verb|len| variable, to keep track of the length of the array.


    \item \textbf{Control flow: for-loops} \\
    The for-loop in Python and C works a little differently. Python employs what is sometimes dubbed a 'for-each' loop, where it goes through every item in an iterable (i.e. 'for each item in this iterable...').

    Meanwhile, C for-loops looks like the following: \\
    \begin{typewriter}
        for (init\_clause ; cond\_expr ; iteration\_expr)
    \end{typewriter} \\
    where 
    \begin{itemize}
    \item \verb|init_clause| specifies the initial condition
    \item \verb|cond_expr| specifies when the loop should end
    \item \verb|iteration_expr| specifies what happens at the end of each loop
    \end{itemize}  
    
    One of the most common manners this loop is used can be seen in the example above (line 8 of the C code), which is strikingly similar to using Python's for-loop with the \verb|range()| function (line 3 of the Python code). However, do note that C doesn't restrict \verb|cond_expr| to be related to the \verb|init_clause| or \verb|iteration_expr| - you're free, for example, to use variables not declared in the \verb|init_clause|.

    \item \textbf{Control flow: if, else if, else; continue and break} \\
    These works similar to Python. Do note that the condition needs to be in parantheses, and that there is no \verb|elif| keyword -- instead, write \verb|else if|.
    
    \item \textbf{Control flow: while loops}\\
    While not present in the example, C has while-loops behaving similarly to Python's. 
\end{itemize}

At this point, we have covered the very basics of the C programming language. You should know how to declare and initialize variables, use the control flow mechanisms, as well as write functions. 
% Below are some simple exercises you can attempt to try out what has been explored above: \\

% \begin{exercise}
% Fibonacci numbers \\
% Write a program that prints out the first 10 Fibonacci numbers.\\
% \end{exercise}

% \begin{exercise}
% Palindrome number \\

% \end{exercise}