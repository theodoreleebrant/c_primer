
\chapter{Basic C Syntax}
    \section{Variable Declaration}
    In C, variables need to be declared before it is used. A variable declaration includes declaring the type of the variable. The following example declares a variable \mintinline{c}{x} with the type \mintinline{c}{int}, a variable \mintinline{c}{y} with the type \mintinline{c}{float}, and a variable \mintinline{c}{z} with the type \mintinline{c}{int}:
    \begin{minted}{c}
        int x;
        float y;
        int z;
    \end{minted}
    After the variable have been declared, values can be assigned to it using the \verb|=| operator:
    \begin{minted}{c}
        x = 2;
        y = 3.0;
        z = x * 5;
    \end{minted}
    Additionally, you can also combine variable assignment and declaration in one line:
    \begin{minted}{c}
        int b = 10;
        float f = 3.1415;
    \end{minted}

    [Java]
    The basics of variable declaration and assignment in C has no major difference from Java's own variable declaration and assignment.
    
    [Python]
    In Python, a variable is created at the point where you assign a value to it, e.g. writing \mintinline{python}{q=1.602} in Python will create the variable q with the value 1.602. 
    
    If you try to assign a value to a variable which is not yet declared in C, the compiler would refuse to compile your code. Additionally, once you declare a variable to be of a particular type, you cannot change its type: If the variable x is declared of type double, and you assign “x = 3;”, then x will actually hold the floating-point value 3.0 rather than the integer 3. You can, if you wish, imagine that x is a box that is capable only of holding double values; if you attempt to place something else into it (like the int value 3), the compiler converts it into a double so that it will fit. %C for Py Devs

    %summarybox
    % - Variables need to be declared before usage, including assignment
    % - Variable declaration and assignment can be done in one line
    % - Declaring variable also assigns the type of the variable, which cannot be changed

    %exercise
    % exercise 1: (for compiler warning) use a variable without decl; e.g. int a = 2; int b = 3; c = a * b;
    % exercise 2: ???
    
    \section{Statements and Blocks}
    [Py] Whitespace whitespace whitespace...
    Statement is endl, block is indentation

    [C/Java] End statements with semicolons, blocks are demarcated by  \{\}
    
    \section{Control Flow}
        \subsection{Conditionals}
        if / switch
        \subsection{Iterations}
        while / for / do-while
        break, continue
        \subsection{Others}
        goto considered harmful

\chapter{Basic C Features}
    \section{Comments}
    Single-line comment with //, multiline comment with /* */ 
    
    [Py] \#
    
    [Java] // only
    
    Trivia: Prior to C99, C only supports /* */

    
    \section{Primitive Types}
    int, short, long, double, float, char

    [Java] different char, no boolean

    [Python] uhh good luck lol 

    [Both] No boolean. 0 is false, non-zero is true. See: truthy values in Python

    The special case of void
    
    \section{Operators}
    the operators and precedence table.

    [Py] pow, and, or, not. precedence of not.

    [Java] + is not string concat. Strings are not even primitive types

    Trivia: assignment is an operator, returns the value assigned. Can lead to bad style / unreadable code if chained too much!
    
    \section{Arrays}
    basic syntax; 0-index. initialization.

    [Py] static length.
    
    remember, no way to access length of array; neither is there bounds checking.
    
    
    \section{Strings}
    Strings is just char array ended with NUL (ASCII 0) - this makes the length of the array = length of string + 1 at least. Mutable.
        \subsection{String Formatting}
        escape characters, percent-character
    
    \section{Structs and Custom Type Definition}
    We want to make more expressive types; e.g. something that represents a point with two integer x and y coordinate. Java / Py will resort to classes. C - no such thing. We can use structs, which are types composed of other types. 

    declaration syntax; member; tag; access syntax

    % skipping unions

    typedef structs;
    other typedefs;

\chapter{Functions}
    \section{Function Parameters and Arguments}
    \section{Function Declarations}
    Forward declaration / Function prototypes
    \section{The main Function}

\chapter{Libraries and Linking}
    \section{Preprocessor}
    \section{System Libraries}
    \section{Header Files}
    \section{Compiling and Linking Multiple Files}
    \section{The Makefile}

\chapter{Pointers and Memory}
    \section{Variables, Addresses, and Pointers}
    \section{Passing Arguments by Reference}
    \section{Dynamic Memory Allocation}
    \section{Function Pointers}

\chapter{Debugging}
    \section{GDB}
    \section{Valgrind}