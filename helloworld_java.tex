% Cut-out parts from the Java section of Ch.2

\chapter{Hello, World!}
When developing programs using a programming language, knowing how to execute your program is an important process. For example, when you have a Python file called \verb|my_program.py|, you might run it by using 
\begin{minted}{shell}
python3 my_program.py
\end{minted}
in your terminal. Conversely, if you use Java to write your code (say, \verb|Main.java|), you might run it by using
\begin{minted}{shell}
javac Main.java
\end{minted}
which produces the class file called \verb|Main.class|, which you can then 'run' by using the
\begin{minted}{shell}
java Main
\end{minted}
command.

This roughly shows the difference between executing interpreted and compiled languages. In the case of the Python example above, we have the \textit{interpreter} (\verb|python3|) reading the user's code line-by-line and executing it directly. Conversely, in the case of the Java example, executing the code involves two steps: running the \textit{compiler} (\verb|javac|) to make the class file before being able to run it.

C is a compiled language similar to Java. To execute a code written in C (e.g. \verb|my_program.c|), we first invoke the compiler:
\begin{minted}{bash}
gcc my_program.c
\end{minted}
which creates the executable file called \verb|a.out|, which we can then run by using
\begin{minted}{bash}
./a.out
\end{minted}

Note: As it is executing the program, the computer has no idea that a.out was just created from some C program: It is simply blindly executing the code found within the a.out file, just as it blindly executes the code found within the gcc file in response to the first command. % from C for Python Devs

Exercise 1:
Create a file called \verb|hello.c| with the following content:
\begin{minted}{c}
#include <stdio.h>
int main(int argc, char *argv[]) {
    printf("Hello world!\n");
}
\end{minted}
Compile and run the program.