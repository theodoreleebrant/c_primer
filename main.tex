\documentclass[oneside]{book}
\usepackage{graphicx} % Required for inserting images
\usepackage{palatino}

\usepackage{setspace}
\onehalfspacing

\usepackage{minted}
\usemintedstyle{pastie}

\setlength\parindent{0pt} %noindent


\title{Quick Start to C: A Primer}
\author{Theodore Leebrant}
\date{June 2024}

\begin{document}

% \maketitle
\begin{titlepage}
	\centering
	{\Huge \textsc{Quick Start to C: A Primer}\par}
	\vspace{1.5cm}
	{\large \itshape Theodore Leebrant\par}
	\vfill
	{\Large Licensed under a\par
	Creative Commons Attribution-ShareAlike 4.0\par
    International License}

	\vfill


	{\large June 2024\par}
\end{titlepage}

\tableofcontents

\setlength{\parskip}{12pt}

\chapter{Introduction}
This primer is intended for a reader who has a working knowledge of a high-level programming language and want to start to learn how to program in C. It is by no means a substitute to an "Introduction to Programming" course, but rather something meant to bridge the gap between the transferable knowledge of programming concepts the reader have learnt prior, to writing code in C. 

As a primer, this book is not meant to be a complete guide to the C programming language as well - there are features of the language which will be skipped or glossed over; the book is meant to give the readers a quick start to start writing code in C. As a feature, this primer will contain some examples and highlights on the differences between C and Java / Python 3.

This work builds and adapts upon the contents from "C for Python programmers" by Carl Burch, used under CC BY-SA 3.0 US and "C for Java Programmers" by George Ferguson, used under CC BY-SA 4.0.
% TODO: Add hyperlinks
% http://www.cs.toronto.edu/~patitsas/cs190/c_for_python.html
% https://www.cs.rochester.edu/u/ferguson/csc/c/c-for-java-programmers.pdf


\chapter{Running Your Code}
When developing programs using a programming language, knowing how to execute your program is an important process. For example, when you have a Python file called \verb|my_program.py|, you might run it by using 
\begin{minted}{shell}
python3 my_program.py
\end{minted}
in your terminal. Conversely, if you use Java to write your code (say, \verb|Main.java|), you might run it by using
\begin{minted}{shell}
javac Main.java
\end{minted}
which produces the class file called \verb|Main.class|, which you can then 'run' by using the
\begin{minted}{shell}
java Main
\end{minted}
command.

\pagebreak

This roughly shows the difference between executing interpreted and compiled languages. In the case of the Python example above, we have the \textit{interpreter} (\verb|python3|) reading the user's code line-by-line and executing it directly. Conversely, in the case of the Java example, executing the code involves two steps: running the \textit{compiler} (\verb|javac|) to make the class file before being able to run it.

C is a compiled language similar to Java. To execute a code written in C (e.g. \verb|my_program.c|), we first invoke the compiler:
\begin{minted}{bash}
gcc my_program.c
\end{minted}
which creates the executable file called \verb|a.out|, which we can then run by using
\begin{minted}{bash}
./a.out
\end{minted}

Note: As it is executing the program, the computer has no idea that a.out was just created from some C program: It is simply blindly executing the code found within the a.out file, just as it blindly executes the code found within the gcc file in response to the first command. % from C for Python Devs

Exercise 1:
Create a file called \verb|hello.c| with the following content:
\begin{minted}{c}
#include <stdio.h>
int main(int argc, char *argv[]) {
    printf("Hello world!\n");
}
\end{minted}
Compile and run the program.

\chapter{Basic C Syntax}
    \section{Variable Declaration}
    In C, variables need to be declared before it is used. A variable declaration includes declaring the type of the variable. The following example declares a variable \mintinline{c}{x} with the type \mintinline{c}{int}, a variable \mintinline{c}{y} with the type \mintinline{c}{float}, and a variable \mintinline{c}{z} with the type \mintinline{c}{int}:
    \begin{minted}{c}
        int x;
        float y;
        int z;
    \end{minted}
    After the variable have been declared, values can be assigned to it using the \verb|=| operator:
    \begin{minted}{c}
        x = 2;
        y = 3.0;
        z = x * 5;
    \end{minted}
    Additionally, you can also combine variable assignment and declaration in one line:
    \begin{minted}{c}
        int b = 10;
        float f = 3.1415;
    \end{minted}

    [Java]
    The basics of variable declaration and assignment in C has no major difference from Java's own variable declaration and assignment.
    
    [Python]
    In Python, a variable is created at the point where you assign a value to it, e.g. writing \mintinline{python}{q=1.602} in Python will create the variable q with the value 1.602. 
    
    If you try to assign a value to a variable which is not yet declared in C, the compiler would refuse to compile your code. Additionally, once you declare a variable to be of a particular type, you cannot change its type: If the variable x is declared of type double, and you assign “x = 3;”, then x will actually hold the floating-point value 3.0 rather than the integer 3. You can, if you wish, imagine that x is a box that is capable only of holding double values; if you attempt to place something else into it (like the int value 3), the compiler converts it into a double so that it will fit. %C for Py Devs

    %summarybox
    % - Variables need to be declared before usage, including assignment
    % - Variable declaration and assignment can be done in one line
    % - Declaring variable also assigns the type of the variable, which cannot be changed

    %exercise
    % exercise 1: (for compiler warning) use a variable without decl; e.g. int a = 2; int b = 3; c = a * b;
    % exercise 2: ???
    
    \section{Statements and Blocks}
    [Py] Whitespace whitespace whitespace...
    Statement is endl, block is indentation

    [C/Java] End statements with semicolons, blocks are demarcated by  \{\}
    
    \section{Control Flow}
        \subsection{Conditionals}
        if / switch
        \subsection{Iterations}
        while / for / do-while
        break, continue
        \subsection{Others}
        goto considered harmful

\chapter{Basic C Features}
    \section{Comments}
    Single-line comment with //, multiline comment with /* */ 
    
    [Py] \#
    
    [Java] // only
    
    Trivia: Prior to C99, C only supports /* */

    
    \section{Primitive Types}
    int, short, long, double, float, char

    [Java] different char, no boolean

    [Python] uhh good luck lol 

    [Both] No boolean. 0 is false, non-zero is true. See: truthy values in Python

    The special case of void
    
    \section{Operators}
    the operators and precedence table.

    [Py] pow, and, or, not. precedence of not.

    [Java] + is not string concat. Strings are not even primitive types

    Trivia: assignment is an operator, returns the value assigned. Can lead to bad style / unreadable code if chained too much!
    
    \section{Arrays}
    basic syntax; 0-index. initialization.

    [Py] static length.
    
    remember, no way to access length of array; neither is there bounds checking.
    
    
    \section{Strings}
    Strings is just char array ended with NUL (ASCII 0) - this makes the length of the array = length of string + 1 at least. Mutable.
        \subsection{String Formatting}
        escape characters, percent-character
    
    \section{Structs and Custom Type Definition}
    We want to make more expressive types; e.g. something that represents a point with two integer x and y coordinate. Java / Py will resort to classes. C - no such thing. We can use structs, which are types composed of other types. 

    declaration syntax; member; tag; access syntax

    % skipping unions

    typedef structs;
    other typedefs;

\chapter{Functions}
    \section{Function Parameters and Arguments}
    \section{Function Declarations}
    Forward declaration / Function prototypes
    \section{The main Function}

\chapter{Libraries and Linking}
    \section{Preprocessor}
    \section{System Libraries}
    \section{Header Files}
    \section{Compiling and Linking Multiple Files}
    \section{The Makefile}

\chapter{Pointers and Memory}
    \section{Variables, Addresses, and Pointers}
    \section{Passing Arguments by Reference}
    \section{Dynamic Memory Allocation}
    \section{Function Pointers}

\chapter{Debugging}
    \section{GDB}
    \section{Valgrind}

\end{document}
